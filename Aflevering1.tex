\documentclass[a4paper]{article}
\usepackage[utf8]{inputenc}
\usepackage[danish]{babel}
\usepackage{amsmath}

\title{Boblesortering}
\author{Emir Sircic s164413}
\date{\today}


\begin{document}
\maketitle

\section{Introduktion}

Boblesortering (\emph{eng. bubble sort}) er en populær \emph{sorteringsalgoritme} og er en af de simpleste algoritmer at forstå og implementere. Dog er den ikke en særlig effektiv sorteringsalgoritme\footnote{Mere om dette i "Algoritmer og Datastrukturer 1"} ; hverken for store eller små lister, og den anvendes sjældent i praksis. Boblesortering sorterer, som navnet antyder, elementerne i en liste ved at \emph{boble} hvert element gennem listen til sin rette plads i listen.

\subsection{Pseudokode}
Wikipedia \cite{Wikipedia} giver følgende pseudokode for boblesortering.
\begin{verbatim}
proceture bubbleSort( A : list of sortable items ) defines as :
  do 
    swapped := false
    for each i in 0 to length(A) - 2 inclusive do:
      if A[i] > A[i+1] then
        swap( A[i], A[i+1] )
        swapped := true
      end if
    end for
  while swapped
end procedure
\end{verbatim}
En illustration af en kørsel af boblesortering fra Wikipedia kan ses på figur 1.
\section{Analyse af boblesortering}
\section{Videre læsning}
\end{document}